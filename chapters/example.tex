% chapter1 example

\chapter{Main chapter heading level 1}

\section{Section heading level 2}

\subsection{Section heading level 3}

Referenz auf Glossar: \gls{cross} 

Referenz auf Glossar: \gls{HFU}

Element1\index{Element1} und Element2\index{Element2} erscheinen im Stichwortverzeichnis.

\grqq Das ist ein direktes Zitat \parencite[S. 52-54]{example}\glqq{} und das ist ein indirektes Zitat \parencite{examplewebsite}.

Mathematik-Formel mit Einheiten:

\begin{center}
    \vspace*{-12pt}
    \[ b \geq \frac{2 \cdot T_{eq} \cdot S}{\pi \cdot d^2 \cdot \tau_{KW}} = \frac{2 \cdot 1,2 \cdot \SI[]{1}[]{\N\m} \cdot 2,5}{\pi \cdot \SI[]{6}[]{\mm}^2 \cdot 0,4 \cdot \SI[]{20}[]{\N\per\mm\squared}} \approx \SI[]{6,63}[]{\mm} \]
\end{center}

Abbildung \ref{fig:example1} zeigt die Standarddarstellung eines Bildes.

% image
\begin{figure}[h]
    \centering
    \includegraphics[width=0.6\textwidth]{images/example.png}
    \caption{Beschreibung Bild 1}
    \label{fig:example1}
\end{figure}

Abbildung \ref{fig:example2} und \ref{fig:example3} zeigen zwei Bilder, die nebeneinander dargestellt werden.

\begin{figure}[h]
    \centering
    \begin{minipage}[t]{.5\textwidth}
        \centering
        \captionsetup{width=.8\linewidth}
        \includegraphics[width=0.8\linewidth]{images/example.png}
        \caption{Beschreibung Bild 2}
        \label{fig:example2}
    \end{minipage}%
    \begin{minipage}[t]{0.5\textwidth}
        \centering
        \captionsetup{width=.8\linewidth}
        \includegraphics[width=0.8\linewidth]{images/example.png}
        \caption{Beschreibung Bild 3}
        \label{fig:example3}
    \end{minipage}
\end{figure}

\FloatBarrier

Abbildung \ref{fig:example4} zeigt ein Bild im Landschaftsformat.

\begin{sidewaysfigure}[h] 
    \centering
    \includegraphics[width=0.5\linewidth]{images/example.png}
    \caption{Beschreibung Bild 4}
    \label{fig:example4}
\end{sidewaysfigure}

\FloatBarrier

Aufzählung:

\begin{itemize}
    \vspace*{-12pt}
    \item Element 1
    \item Element 2
    \item Element 3
    \vspace*{-12pt}
\end{itemize}

\FloatBarrier

Tabelle \ref{tab:Tabelle} zeigt eine Tabelle mit Zeilenumbruch.

\begin{table}[h]
    \renewcommand\arraystretch{1.5} % increase or reduce horizonatl space between rows
    % \renewcommand{\tabcolsep}{3pt} % reduce vertical space between columns
    \centering
    \caption{Beschreibung Tabelle 1} % To place a caption above a table
    \begin{tabular}{l*{2}{>{\raggedright\arraybackslash}p{0.3\linewidth}}} % star notation: *{number of columns}{column-type definition}
    \toprule % Thick line above table
    \textbf{Heading 1} & \textbf{Heading 2} & \textbf{Heading 3} \\
    \midrule
    Some text & Some longer text & This text is even longer and will force a linebreak \\
    \bottomrule % Thick line below table
    \label{tab:Tabelle}
    \end{tabular}
\end{table}%

\FloatBarrier

Tabelle \ref{tab:Tabelle2} zeigt eine weitere Tabelle mit Zeilenumbruch.

\begin{table}[h]	
    \renewcommand\arraystretch{1.5} % extra space between rows
    \centering
    \caption{Beschreibung Tabelle 2}
    \begin{tabular}{>{\raggedright\arraybackslash}p{0.3\linewidth}>{\raggedright\arraybackslash}p{0.6\linewidth}}
    \toprule
    \textbf{Heading 1} & \textbf{Heading 2} \\
    \midrule
    Some text &  This text has to be even longer in order to create a linebreak \\
    \bottomrule
    \label{tab:Tabelle2}
    \end{tabular}
    \vspace{-10mm} % Reduce space underneath table
\end{table}%


