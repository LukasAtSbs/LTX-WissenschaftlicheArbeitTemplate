%
%       LateX template for academic papers following HFU guidlines     
%
%       optimised for german language, annotated in English
%       
%       Author: Lukas Walchner
%       Last update: 17.04.2023
%
        
\documentclass[12pt, twoside, openright]{report} % layout for bookbinding

%----------------------------------------------------------------------------------------
%	SIMPLE PACKAGES
%----------------------------------------------------------------------------------------

% page layout
\usepackage[a4paper, left=25mm, right=25mm, top=25mm, bottom=25mm, bindingoffset=15mm]{geometry}

% language packages
\usepackage[utf8]{inputenc} % allows input of national letters
\usepackage[T1]{fontenc} % support for specific German characters
\usepackage[UKenglish, ngerman]{babel} % last languages is the considered the main language

% images
\usepackage{graphicx}
\graphicspath{images} % Folder for images in the template folder

% lipsum text
\usepackage{lipsum}

% math package
\usepackage{amsmath}

% units package
\usepackage{siunitx}
\sisetup{locale = DE}

% chemical formulars
\usepackage{chemformula}

% enumerate package of formating enum items
\usepackage{enumitem}

% reference package
\usepackage{hyperref}

% useful package to format tables
\usepackage{rotating}

% inclusion of pdf pages
\usepackage{pdfpages}

% multi columns
\usepackage{multicol}

% \FloatBarrier command to prevent floats from appearing after a certain point
\usepackage[section]{placeins}

% caption package; allows to set a caption witdh
\usepackage[]{caption}

%----------------------------------------------------------------------------------------
%	Table formatting
%----------------------------------------------------------------------------------------

% package that allows to break lines within cells with fixed width columns
\usepackage{array}

% useful package to format tables
\usepackage{booktabs}

% package that allows you to set the width of a table
\usepackage{tabularx}



%----------------------------------------------------------------------------------------
%	HEADER & FOOTER
%----------------------------------------------------------------------------------------

% header
\usepackage{fancyhdr}
\setlength{\headheight}{25pt}
\pagestyle{fancy}
\renewcommand{\chaptermark}[1]{\markboth{#1}{}} % only chapter name in chaptermark
\fancyhf{} % clear header and footer fields
\fancyhead[LE, RO]{\includegraphics[height=20pt, keepaspectratio]{files/Logo_HFU_42x15mm.jpg}} % page number
\fancyhead[RE, LO]{\normalfont \bfseries \nouppercase{\leftmark}} % title of chapter

\renewcommand{\headrulewidth}{0.4pt}
\fancypagestyle{plain}{} % plain is the default page style for titlepage and chapters in report document class

% footer
\fancyfoot[C]{\normalfont{\nouppercase{\thepage}}} % page number

% blank pages between chapters with no headers
\makeatletter
\def\cleardoublepage{\clearpage\if@twoside \ifodd\c@page\else
    \begingroup
    \mbox{}
    \thispagestyle{empty}
    \newpage
    \if@twocolumn\mbox{}\newpage\fi
    \endgroup\fi\fi}
\makeatother
    

%----------------------------------------------------------------------------------------
%	HEADLINES
%----------------------------------------------------------------------------------------

% headline format
\usepackage{titlesec}

% chapter headline
\titleformat{\chapter} % command
    {\normalfont \large \bfseries} % format
    {\thechapter} % label
    {1em} % horizontal separation between label and title body
    {} % before-code
\titlespacing*{\chapter} % command; * for killing indentation after title
    {0pt} % left
    {0pt} % vertical space before title
    {0pt} % separation between title and text

% section headline
\titleformat{\section} % command
    {\normalfont \bfseries} % format
    {\thesection} % label
    {1em} % horizontal separation between label and title body
    {} % before-code
\titlespacing*{\section} % command; * for killing indentation after title
    {0pt} % left
    {0pt} % vertical space before title
    {0pt} % separation between title and text

% subsection headline
\titleformat{\subsection} % command
    {\normalfont} % format
    {\thesubsection} % label
    {1em} % horizontal separation between label and title body
    {} % before-code
\titlespacing*{\subsection} % command; * for killing indentation after title
    {0pt} % left
    {0pt} % vertical space before title
    {0pt} % separation between title and text


%----------------------------------------------------------------------------------------
%	TOC, GLOSSAIRES, BIBLOGRAPHY, INDEX
%----------------------------------------------------------------------------------------

% customization package for ToC, LoF & LoT
\usepackage[titles]{tocloft}
\setlength{\cftbeforechapskip}{3pt} % linespacing for ToC, LoF & LoT

% glossaries
\usepackage[nopostdot, acronym, toc]{glossaries}
\makenoidxglossaries
\loadglsentries{files/glossary.tex} % loading definition from file glossary.tex
\loadglsentries{files/acronyms.tex} % loading definition from file acronyms.tex

% index package
\usepackage{imakeidx}
\makeindex[columns=2, title=Stichwortverzeichnis]

% bibliography
\usepackage[style=ieee]{biblatex}
\addbibresource{files/references.bib} % bibliography file
\renewcommand*{\UrlFont}{\rmfamily} % url in current font
\DefineBibliographyStrings{ngerman}{ % redefine the command urlseen
  urlseen = {abgerufen am},
}


%----------------------------------------------------------------------------------------
%	TEXT STRUCTURE
%----------------------------------------------------------------------------------------

% definition of front-, main-, backmatter for documentclass report
\makeatletter
\newcommand\frontmatter{%
    \cleardoublepage
    \pagenumbering{Roman}}
    
\newcommand\mainmatter{%
    \cleardoublepage
    \pagenumbering{arabic}}
    
\newcommand\backmatter{%
    \if@openright
    \cleardoublepage
    \else
    \clearpage
    \fi}
\makeatother

% paragraph format
\setlength{\parindent}{0pt} % indentation
\setlength{\parskip}{1em} % paragraph separation

% convenient line spacing package, that excludes captions, footnotes...
\usepackage{setspace}

% numbering package
\usepackage{chngcntr}
\counterwithout{figure}{chapter} % no numbering reset of images with a new chapter
\counterwithout{table}{chapter} % no numbering reset of tables with a new chapter


%----------------------------------------------------------------------------------------
%	DOCUMENT
%----------------------------------------------------------------------------------------

\begin{document}

\onehalfspacing % line spacing 1.5

\setlength{\abovedisplayskip}{0pt} % Space above equations
\setlength{\belowdisplayskip}{0pt} % Spaxe below equations

% titlepage
\begin{titlepage}

  \begin{figure}
      \centering
      \begin{minipage}{.5\textwidth}
        \centering
        \includegraphics[width=.8\linewidth]{files/SBS-FCOL.png}
      \end{minipage}%
      \begin{minipage}{.5\textwidth}
        \centering
        \includegraphics[width=.8\linewidth]{files/Logo_HFU_42x15mm.jpg}
      \end{minipage}
  \end{figure}

  \begin{center}



      \vspace*{1cm}

      {\Large \selectfont {Wissenschaftliche Arbeit}} \\
      {\normalsize \selectfont {in}} \\
      {\Large \selectfont {Studiengang XXXXXXXXXX}}

      {\small \selectfont {Fakultät XXXXXXXXXX}}

      \vspace*{1cm}

      {\huge \selectfont \bfseries {Titel}}

      {\huge \selectfont \bfseries {der}}

      {\huge \selectfont \bfseries {Arbeit}}

      {\Large \selectfont {Untertitel der Arbeit}}

      \vspace*{4cm}

      \begin{tabular}{ l l }
          1. Prüfer: & XXXXXXXXXX\\
          2. Prüfer: & XXXXXXXXXX \\
          Vorgelegt am: & XX.XX.XXXX \\
          Vorgelegt von: & XXXXXXXXXX \\
          & Straße und Hausnummer \\
          & PLZ und Stadt \\
          & E-Mail-Adresse \\
      \end{tabular}

  \end{center}
  
\end{titlepage}

\frontmatter

% Abstract in deutsch (mandatory)
\chapter*{Abstract (deutsch)}
\markboth{Abstract (deutsch)}{} % Abstract to chaptermark
\addcontentsline{toc}{chapter}{Abstract (deutsch)} % Abstract to ToC
% Abstract deutsch

\begin{tabularx}{\linewidth}{@{}>{\bfseries}l@{\hspace{.5em}}X@{}}
    Titel: & XXXXXXXXXX \\
    Verfasser: & XXXXXXXXXX \\
    1. Prüfer: & XXXXXXXXXX \\
    2. Prüfer: & XXXXXXXXXX \\
    Semester: & X \\
\end{tabularx}

\lipsum[1-2]

\begin{tabularx}{\linewidth}{@{}>{\bfseries}l@{\hspace{.5em}}X@{}}
    Schlüsselwörter: & XXXXXXXXXX, XXXXXXXXXX, XXXXXXXXXX  \\
\end{tabularx}




% Abstract in englisch (mandatory)
\chapter*{Abstract (english)}
\markboth{Abstract (english)}{} % Abstract to chaptermark
\addcontentsline{toc}{chapter}{Abstract (english)} % Abstract to ToC

\begin{otherlanguage}{UKenglish}

\begin{tabularx}{\linewidth}{@{}>{\bfseries}l@{\hspace{.5em}}X@{}}
    Title: & XXXXXXXXXX \\
    Author: & XXXXXXXXXX \\
    1. Examiner: & XXXXXXXXXX\\
    2. Examiner: & XXXXXXXXXX \\
    Semester: & X \\
\end{tabularx}

\lipsum[1-2]

\begin{tabularx}{\linewidth}{@{}>{\bfseries}l@{\hspace{.5em}}X@{}}
    Keywords: & XXXXXXXXXX, XXXXXXXXXX, XXXXXXXXXX \\
\end{tabularx}

\end{otherlanguage}

% Eidesstattliche Erklärung (mandatory)
\chapter*{Eidesstattliche Erklärung}
\markboth{Eidesstattliche Erklärung}{} % Eidesstattliche Erklärung to chaptermark
\addcontentsline{toc}{chapter}{Eidesstattliche Erklärung} % Eidesstattliche Erklärung to ToC
% declaration

Ich erkläre hiermit an Eides statt, dass ich die vorliegende Arbeit selbständig und ohne unzulässige fremde Hilfe angefertigt habe. 

Alle verwendeten Quellen (Literatur, Internet) sind im Literaturverzeichnis vollständig zitiert.


\vspace{4cm}

\rule{12cm}{0.4pt}

Ort, Datum, Unterschrift

% Sperrvermerk (optional)
\chapter*{Sperrvermerk}
\markboth{Sperrvermerk}{} % Sperrvermerk to chaptermark
\addcontentsline{toc}{chapter}{Sperrvermerk} % Sperrvermerk to ToC
% Confidential Clause

Die vorliegende wissenschaftliche Arbeit beinhaltet vertrauliche Informationen und Daten. 

Sie darf nur vom Erst- und Zweitgutachter sowie berechtigten Mitgliedern des Prüfungsausschusses eingesehen werden.
Eine Vervielfältigung und Veröffentlichung ist auch auszugsweise nicht erlaubt. 

Dritten darf diese Arbeit nur mit der ausdrücklichen Genehmigung des Verfassers und Unternehmens zugänglich gemacht werden.


% Introduction (optional)
\chapter*{Vorwort}
\markboth{Vorwort}{} % Introduction to chaptermark
\addcontentsline{toc}{chapter}{Vorwort} % Introduction to ToC
% indroduction

\lipsum[1-6]

% Inhaltsverzeichnis (mandatory)
\tableofcontents    
\addcontentsline{toc}{chapter}{Inhaltsverzeichnis} % chapter Inhaltsverzeichnis zu ToC

% Abbildungsverzeichnis (mandatory)
\listoffigures
\addcontentsline{toc}{chapter}{Abbildungsverzeichnis} % Abbildungsverzeichnis zu ToC

% Tabellenverzeichnis (mandatory)
\listoftables
\addcontentsline{toc}{chapter}{Tabellenverzeichnis} % Tabellenverzeichnis zu ToC

% Glossary (optional)
\printnoidxglossary % type=main

% List of abbreviations (optional)
\printnoidxglossary[type=\acronymtype] % type=acronym


\mainmatter

%----------------------------------------------------------------------------------------
%	CONTENT
%----------------------------------------------------------------------------------------

% Chapter 1
% chapter1 example

\chapter{Main chapter heading level 1}

\section{Section heading level 2}

\subsection{Section heading level 3}

Referenz auf Glossar: \gls{cross} 

Referenz auf Glossar: \gls{HFU}

Element1\index{Element1} und Element2\index{Element2} erscheinen im Stichwortverzeichnis.

\grqq Das ist ein direktes Zitat \parencite[S. 52-54]{example}\glqq{} und das ist ein indirektes Zitat \parencite{examplewebsite}.

Mathematik-Formel mit Einheiten:

\begin{center}
    \vspace*{-12pt}
    \[ b \geq \frac{2 \cdot T_{eq} \cdot S}{\pi \cdot d^2 \cdot \tau_{KW}} = \frac{2 \cdot 1,2 \cdot \SI[]{1}[]{\N\m} \cdot 2,5}{\pi \cdot \SI[]{6}[]{\mm}^2 \cdot 0,4 \cdot \SI[]{20}[]{\N\per\mm\squared}} \approx \SI[]{6,63}[]{\mm} \]
\end{center}

Abbildung \ref{fig:example1} zeigt die Standarddarstellung eines Bildes.

% image
\begin{figure}[h]
    \centering
    \includegraphics[width=0.6\textwidth]{images/example.png}
    \caption{Beschreibung Bild 1}
    \label{fig:example1}
\end{figure}

Abbildung \ref{fig:example2} und \ref{fig:example3} zeigen zwei Bilder, die nebeneinander dargestellt werden.

\begin{figure}[h]
    \centering
    \begin{minipage}[t]{.5\textwidth}
        \centering
        \captionsetup{width=.8\linewidth}
        \includegraphics[width=0.8\linewidth]{images/example.png}
        \caption{Beschreibung Bild 2}
        \label{fig:example2}
    \end{minipage}%
    \begin{minipage}[t]{0.5\textwidth}
        \centering
        \captionsetup{width=.8\linewidth}
        \includegraphics[width=0.8\linewidth]{images/example.png}
        \caption{Beschreibung Bild 3}
        \label{fig:example3}
    \end{minipage}
\end{figure}

\FloatBarrier

Abbildung \ref{fig:example4} zeigt ein Bild im Landschaftsformat.

\begin{sidewaysfigure}[h] 
    \centering
    \includegraphics[width=0.5\linewidth]{images/example.png}
    \caption{Beschreibung Bild 4}
    \label{fig:example4}
\end{sidewaysfigure}

\FloatBarrier

Aufzählung:

\begin{itemize}
    \vspace*{-12pt}
    \item Element 1
    \item Element 2
    \item Element 3
    \vspace*{-12pt}
\end{itemize}

\FloatBarrier

Tabelle \ref{tab:Tabelle} zeigt eine Tabelle mit Zeilenumbruch.

\begin{table}[h]
    \renewcommand\arraystretch{1.5} % increase or reduce horizonatl space between rows
    % \renewcommand{\tabcolsep}{3pt} % reduce vertical space between columns
    \centering
    \caption{Beschreibung Tabelle 1} % To place a caption above a table
    \begin{tabular}{l*{2}{>{\raggedright\arraybackslash}p{0.3\linewidth}}} % star notation: *{number of columns}{column-type definition}
    \toprule % Thick line above table
    \textbf{Heading 1} & \textbf{Heading 2} & \textbf{Heading 3} \\
    \midrule
    Some text & Some longer text & This text is even longer and will force a linebreak \\
    \bottomrule % Thick line below table
    \label{tab:Tabelle}
    \end{tabular}
\end{table}%

\FloatBarrier

Tabelle \ref{tab:Tabelle2} zeigt eine weitere Tabelle mit Zeilenumbruch.

\begin{table}[h]	
    \renewcommand\arraystretch{1.5} % extra space between rows
    \centering
    \caption{Beschreibung Tabelle 2}
    \begin{tabular}{>{\raggedright\arraybackslash}p{0.3\linewidth}>{\raggedright\arraybackslash}p{0.6\linewidth}}
    \toprule
    \textbf{Heading 1} & \textbf{Heading 2} \\
    \midrule
    Some text &  This text has to be even longer in order to create a linebreak \\
    \bottomrule
    \label{tab:Tabelle2}
    \end{tabular}
    \vspace{-10mm} % Reduce space underneath table
\end{table}%




% ...

%----------------------------------------------------------------------------------------
%	BIBLOGRAPHY, INDEX, APPENDIX
%----------------------------------------------------------------------------------------


% Bibliography (mandatory)
\begingroup
\singlespacing % 1.0 linespacing within one entry
\setlength\bibitemsep{6pt} % 1.5 linespacing between entries
\printbibliography[title=Literaturverzeichnis]
\addcontentsline{toc}{chapter}{Literaturverzeichnis} % Biblography to ToC
\endgroup

% Index (optional)
\printindex

% Anhang (optional)
\chapter*{Anhang A}
\markboth{Anhang A}{} % Anhang A to chaptermark
\addcontentsline{toc}{chapter}{Anhang A} % Anhang A to ToC


\end{document}

